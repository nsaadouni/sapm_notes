
\documentclass{memoir} 
\begin{document}

\chapter{Analysis (Architectural Evaluation)}
\section{Aids}
Why we Evaluate:
\begin{itemize}
\item to inform decision making
\item to	inform	decision	making	or	to	allow	
progression	to	later	stages	in	the	process
\end{itemize}

There are three types of Evaluation:


\begin{enumerate}
\item \textbf{Evaluation By Designer:} The	consequences	of	the	decision	regulate	
how	much	effort	to	put	into	the	process	–	
more	importance	means	more	effort	in	
evaluation. Try	to	use	iterative	approaches	that	get	
deeper	in	order	to	eliminate	unpromising	
alternatives	early.  Don’t	strive	for	perfection,	good	enough	for	
the	context	is	usually	enough.	
\item \textbf{Peer Evaluation:} Fix	on	the	QAs	to	consider	as	part	of	the	review	–	
may	be	determined	by	the	process	or	the	
business	case. The	architect	presents	the	architecture	to	the	
reviewers	–	questions	are	for	information.	The	review	is	driven	by	the	relevant	scenarios	–	
the	architect	talks	the	review	team	through	a	
scenario	demonstrating	the	architecture	meets	
the	requirements	captured	in	the	scenario. The	outcome	is	a	list	of	potential	issues	with	
actions:	fix,	mi.gate,	tolerate,	...	
\item \textbf{External Evaluation:} Means	to	bring	in	additional	expertise. May	represent	some	stakeholder	interests. More	expensive	and	difficult	to	organise	so	
this	will	often	correspond	to	some	major	
hurdle	in	the	process.	
\end{enumerate}

When we evaluate shit we must consider \textbf{Contextual Factors} as well.

These may include, but are not limited to:
\begin{itemize}
\item What	artifacts	are	available?	
\item Who	sees	the	results	of	the	review?	
\item Who	performs	the	evaluation?	
\item Which	stakeholders	will	par.cipate?	
\item How	does	the	evalua.on	relate	to	business	
goals	of	the	system?	
\end{itemize}

In summary, The	larger	and	more	complex	the	system	the	
more	likely	you	are	to	have	done	explicit	
architectural	design	and	any	design	should	be	
evaluated.
\end{document}